\NeedsTeXFormat{LaTeX2e}
\documentclass[a4paper,twocolumn]{article}
\usepackage{epsf}

\parindent 0cm

\title{yaccviso --- a tool for visualizing yacc grammars}
\author{Leon Aaron Kaplan\\{\tt aaron@lo-res.org}}


\begin{document}
\maketitle

	This work was originaly performed during a ``student praktikum'' at the
	Institute for Computer Languages, Technical University of Vienna,
	Nov. 1997 - Jan. 1998. 


\tableofcontents

\section{Overview}

Convention: whenever We refer to ``yacc'' in this paper, we mean both bison 
	and yacc.

A general knowledge of yacc is assumed.

yaccviso is a tool for visualizing the dependencies of non-terminal and
terminal symbols in a yacc grammar. Thus it should help the person
developing a compiler in yacc to gain a fast overview over his or her grammar
file. The idea we had in mind while developing yaccviso is that the 
person developing his compiler should be able to quickly generate a huge
postscript poster of his or her grammar and have it attached to his wall
so that whenever questions arise all he or she has to do is to turn around
and take a look at the poster.


\section{Syntax}

Presently the invocation syntax is simple:

{\tt yaccviso} [{\tt -h}] [{\sl debugoptions}] {\sl inputfile}

where {\tt -h} prints out a usage message and where {\sl debugoptions} 
are optional and are one of 
\begin{itemize}
	\item {\tt -d DDEBUG}, turns on all debugging messages
	\item {\tt -d DFATAL}, only fatal error messages are printed
	\item {\tt -d DSCANDBG}, debug messages while scanning are printed
	\item {\tt -d DPARSEDBG}, debug messages while parsing
	\item {\tt -d DSEMANALY}, debug messages during sematic analysis
	\item {\tt -d DCODEGEN}, debug messages during output code generation
	\item {\tt -d DSYMTAB}, debug messages concering symbol tables
	\item {\tt -d DIO}, debug messages during input output
	\item {\tt -d DWARNING}, warning messages
\end{itemize}

The defaults for debugging are: {\tt -d DFATAL} and {\tt -d DWARNING}. Other
debug messages should be used with care as they generate a lot of output 
on stderr.

If no input file is given, yaccviso will try to read from stdin.

yaccviso will generate two output files:
{\tt depgraph.vcg} and {\tt depgraph.dot} the former is an input file for the
VCG tool, the latter an input file for the dot tool.

\section{How to interpret the dot output}
The file {\tt depgraph.dot} when fed into the dot(1) program via the line
{\tt dot -Tps -o {\sl outputfilename} depgraph.dot} will be laid out by 
dot(1) in the following fashion:

\begin{enumerate}
	\item terminal symbols will be drawn in color (I used gold, but 
	you can search and replace that with anything you wish of course, or 
	modify the corresponding line in the file {\tt vcg.c} - search for 
	``gold'')
	\item a production consisting of a single line such as 
	{\sl A $\qquad : \qquad$ A1 A2 A3 $\ldots$} will be printed
	like 
	\vskip 0.5cm
	\begin{center}
	\epsfxsize=7cm \epsfbox{dotoutput1.eps}
	\end{center}

	where the ``A'' means that rule {\sl A} depends on the boxes below and
	the ``1'' means that all the boxes below are terminal or
	non terminal symbols on the right hand side of {\em line one} as they 
	appear in the order (i.e. A1 is before A2, etc.). Note the self edge
	in the above example. 
	\item a rule consisting of multiple lines / productions are drawn 
	similar to
	\vskip 0.5cm
	\begin{center}
	\epsfxsize=7cm \epsfbox{dotoutput2.eps}
	\end{center}

	where the ``1'' and ``2'' refer to line one and line two. Note that it
	is possible that a rule can reference the same non terminal or 
	terminal symbol on the right hand side {\em in different lines}.
	For example: \\
	{\sl B $\qquad : \qquad$ C D E} \\
	{\sl \phantom{B} $\qquad \phantom{:} \qquad$ C F G;} \\
	C is used in both lines. 

	This scheme - every line of a rule spawns a box in which subboxes
	are used to point to the corresponding first, second, etc. dependency
	allows for a clear distinction of dependencies.

	Here I have to thank Axel Belifante of the University of Twente, 
	Holland for his usefull code examples.

\end{enumerate}

\section{Functionality}

yaccviso caries out the fillowing steps when given an input file:
\begin{enumerate}
\item	scanning: at this step comments are eliminated and ANSI C code
	as well as preprocessor code which can appear within \%\{ ... \%\}
	sections or within actions is passed to the parser as a specific
	semantic value (yylval). For this purpose a special mini-scanner
	was written in plain C. Thus the normal scanner interleaves its
	operation with this mini scanner. The normal scanner (written in flex)
	will call the mini scanner whenever neccessary. This way we achieve
	a seperation of languages and we don't have to scan ANSI C, 
	preproccesor cammands and yacc files all in one flex scanner.
	The scanner builds up a symbol table.
\item	parsing: accompanied in the distribution you will find a 
	stripped yacc grammar file for parsing yacc files (new\_grammar.txt).
	This grammar is the base for our parser. In addition the parser 
	performs strict syntax error checking and will print out
	meaningful error messages (cf. follow set, \cite{ASU}). Finally
	it controls the construction of a parse tree.
\item	semantic analysis: the parse tree is traversed and neccessary 
	information is extracted and inserted into a hash table with external
	extensions which in turn holds the information about the dependencies.
\item	code emitter: a collection of a few routines which will search
	the dependencies hash table and emit the dependency information
	in two suitable languages: VCG and dot (depgraph.vcg, depgraph.dot).
\end{enumerate}

\section{Related work}
yacc to dot (yacc2dot) - convert yacc files to dot graph descriptions

Author: Philippe Oechslin ({\tt oechslin\@di.epfl.ch})\\
version: 1.0 date: 3.23.95

This gawk programm produces a graph layout for the Dot program from a 
yacc grammar. However it has many restrictions that our approach does not 
have such as only one line per yacc rule. It is much smaller though. 
yacc2dot is included in the graphviz (i.e. dot) distribution.

\section{Future Research}
In future versions we want to add various well known algorithms from the 
field of compiler construction and graph theory such as dominator tree, 
strongly connected components, finding of cycles (i.e. to answer the question
``which rules depend on themselves directly or indirectly'') or subgraph 
starting at a specific rule.

Any suggestions are welcome.



\appendix

\section{known bugs}

\begin{enumerate}
\item {\verb The following does not work:\\
	\#ifdef blabla\\
	\%\}\\
	\#endif\\
   Neither does:\\
	\#ifdef blabla\\
	\}\\
	\#endif\\
   Neither does:\\
	\#ifdef blabla\\
	\{\\
	\#endif\\
   But the following works:\\
	\#ifdef blabla\\
	\{ foobar \}\\
	\#endif\\
\\
   In other words: these types of brackets must always be well formed within
   a \#ifdef environment.
} % end verb
\item please report bugs as you see them accompanied with the input grammar.
\end{enumerate}

\section{References}
\begin{thebibliography}{99}
\bibitem{vcg}VCG, Visualisation of Compiler Graphs, Georg Sander, \\
	{\tt ftp://ftp.cs.uni-sb.de/pub/graphics/vcg}
\bibitem{ASU}Compiler Construction - Prinicples and Design. Aho, Sehti,
	Ullmann, Addison Wesley
\end{thebibliography}




\end{document}
